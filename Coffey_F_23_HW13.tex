\documentclass{article}
\usepackage{graphicx} % Required for inserting images
\usepackage{xcolor}
\usepackage{amsmath}
\usepackage{fancyhdr}
\title{\teftbf PHYS 20323/60323: Fall 2023 - LaTeX Example}
\begin{document}

\begin{center}
    {\normalsize PHYS 20323/60323: Fall 2023 - LaTeX Example}
\end{center}


{\normalsize \textbf {1. The following questions refer to stars in the Table below.}}\\
Note: There may be multiple answers.

{\normalsize{

\begin{center}
\begin{tabular}{|l|c|c|c|c|c|}\hline
Name & Mass & Luminosity & Lifetime & Temperature & Radius & \\\hline
$\eta$Car.    & 60. M{\footnotesize $\odot$}  &  10^6 L{\footnotesize $\odot$} & 8.0 x 10^5 years & & &\\\hline
$\epsilon$Eri.& 6.0 M{\footnotesize $\odot$} &  10^3 L{\footnotesize $\odot$}& & 20,000K  & &\\\hline
$\delta$Scu.  & 2.0 M{\footnotesize $\odot$} & & 5.0 x 10^8 years & & 2R{\footnotesize $\odot$} &\\\hline
$\beta$Cyg.   & 1.3 M{\footnotesize $\odot$} &  3.5 L{\footnotesize $\odot$} & & & &\\\hline
$\alpha$Cen.  & 1.0 M{\footnotesize $\odot$} & & & & 1R{\footnotesize $\odot$} &\\\hline
$\gamma$Del.  & 0.7 M{\footnotesize $\odot$} & & 4.5 x 10^{10} years & 5000K & &\\\hline



\end{tabular}[]
\end{center}

}}

(a) (4 points) Which of these stars will produce a planetary nebula.\\

(b) (4 points) Elements heavier than Carbon will be produced in which stars.\\

{\normalsize{

{\normalsize 2. An electron is found to be in the spin state (in the z-basis)}:    $X = A ({3i\atop4})$ {\fo}\\

(a) (5 points) Determine the possible values of A such that the state is normalized.\\

(b) (5 points) Find the expectation values of the operators {\color{red}S_x} , 
 {\color{purple}S_y} , {\color{orange}S_z} {\color{black} and  \overrightarrow{S}^2}\\

{\color{black} The matrix representations in the z-basis for the components of electron spin operators are given by:}

}}

{\normalsize{

\begin{center}
    {\color{red}S_x = $\overline{h} \atop{2}$ $\biggl($$0\atop{1}$ $1\atop{0} $$\biggl)$};  {\color{purple}S_y = $\overline{h} \atop{2}$ $\biggl($$0\atop{i}$$-i\atop{0}$$\biggl)$};  {\color{orange}S_z = $\overline{h} \atop{2}$ $\biggl($$1\atop{0}$$0\atop{-1}$$\biggl)$}\\
\end{center}


{\color{black}{3. The average electrostatic field in the earth’s atmosphere in fair weather is approximately given:

\begin{center}
    $\overrightarrow{E}$ = $E_0$ $\Bigl($ $Ae^{\alpha z}$  + $Be^{-\beta z}$ $\Bigl)$ $\hat{z}$,}}
\end{center}

}}

{\normalsize{
where A, B, $\alpha$, $\beta$ are positive constants and z is the height above the (locally flat) earth surface.\\

(a) (5 points) Find the average charge density in the atmosphere as a function of height\\

(b) (5 points) Find the electric potential as a function height above the earth.}}

\fancyfoot{}
\pagestyle{fancy}
\renewcommand{\headrulewidth}{}
\fancyfoot[R]{31}
\fancyfoot[L]{Latex Example}





\end{document}
